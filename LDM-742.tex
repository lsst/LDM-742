\documentclass[OPS,toc]{lsstdoc}
% lsstdoc documentation: https://lsst-texmf.lsst.io/lsstdoc.html

% Generated by Makefile
\input{meta}

% Package imports go here.

% Local commands go here.

% If you want glossaries, uncomment:
% \input{aglossary.tex}
% \makeglossaries

\title{Vera C. Rubin Observatory DM Infrastructure Network Verification Document}
% \setDocSubtitle{Optional subtitle}

\author{%
Gabriele Comoretto
}

\setDocRef{LDM-742}
\setDocUpstreamLocation{\url{https://github.com/lsst/LDM-742}}
\date{\vcsDate}
% \setDocCurator{The Curator of this Document}

\setDocAbstract{%
Add abstract text.
}

% Revision history.
% Order: oldest first.
% Fields: VERSION, DATE, DESCRIPTION, OWNER NAME.
% See LPM-51 for version number policy.
\setDocChangeRecord{%
  \addtohist{1}{2020-07-30}{Unreleased.}{Gabriele Comoretto}
}

\begin{document}

\maketitle

% ADD CONTENT HERE
% You can also use the \input command to include several content files.

\appendix

% Include all the relevant bib files.
% https://lsst-texmf.lsst.io/lsstdoc.html#bibliographies
\section{References} \label{sec:bib}
\renewcommand{\refname}{} % Suppress default Bibliography section
\bibliography{local,lsst,lsst-dm,refs_ads,refs,books}

% Make sure lsst-texmf/bin/generateAcronyms.py is in your path
\section{Acronyms} \label{sec:acronyms}
\addtocounter{table}{-1}
\begin{longtable}{p{0.145\textwidth}p{0.8\textwidth}}\hline
\textbf{Acronym} & \textbf{Description}  \\\hline

AP & Alert Production \\\hline
CCD & Charge-Coupled Device \\\hline
CCOB & Camera Calibration Optical Bench \\\hline
DAC & Data Access Center \\\hline
DAQ & Data Acquisition System \\\hline
DB & DataBase \\\hline
DBB & Data Back Bone \\\hline
DM & Data Management \\\hline
DMCS & Data Management Control System \\\hline
DMS & Data Management Subsystem \\\hline
DMS-REQ & Data Management System Requirements prefix \\\hline
DMSR & DM System Requirements; LSE-61 \\\hline
DOE & Department of Energy \\\hline
DOI & Digital Object Identifier \\\hline
DR & Data Release \\\hline
DR1 & Data Release 1 \\\hline
DRP & Data Release Production \\\hline
EFD & Engineering and Facility Database \\\hline
GUI & Graphical User Interface \\\hline
HSC & Hyper Suprime-Cam \\\hline
ICD & Interface Control Document \\\hline
IN2P3 & Institut National de Physique Nucléaire et de Physique des Particules \\\hline
L1 & Lens 1 \\\hline
L2 & Lens 2 \\\hline
L3 & Lens 3 \\\hline
LAN & Local Area Network \\\hline
LDF & LSST Data Facility \\\hline
LDM & LSST Data Management (Document Handle) \\\hline
LSE & LSST Systems Engineering (Document Handle) \\\hline
LSP & LSST Science Platform \\\hline
LSST & Legacy Survey of Space and Time (formerly Large Synoptic Survey Telescope) \\\hline
MTTR & Mean Time To Repair \\\hline
NCSA & National Center for Supercomputing Applications \\\hline
NSF & National Science Foundation \\\hline
OCS & Observatory Control System \\\hline
OS & Operating System \\\hline
OSS & Observatory System Specifications; LSE-30 \\\hline
PMCS & Project Management Controls System \\\hline
PSF & Point Spread Function \\\hline
PVI & Processed Visit Image \\\hline
QC & Quality Control \\\hline
RAID & Redundant Array of Inexpensive Disks \\\hline
SE & System Engineering \\\hline
SLAC & SLAC National Accelerator Laboratory (formerly Stanford Linear Accelerator Center; SLAC is now no longer an acronym) \\\hline
SP & Survey Performance \\\hline
US & United States \\\hline
WCS & World Coordinate System \\\hline
\end{longtable}

% If you want glossary uncomment below and comment out the two lines above.
% \printglossaries

\end{document}
